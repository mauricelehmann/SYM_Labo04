\input ./tools/pandoc2context/pandoc-xhtml.tex
\input ./setup.tex

\startluacode
    function xml.functions.processCODE(t)
        buffers.assign("typesetcode",
        "\\startCODE[highlight={" ..
          tostring(xml.attribute(t,"..[tag()='pre'","data-lines"))
              .. "}]\n" .. tostring(xml.text(t)) .. "\n\\stopCODE")
        context.getbuffer { "typesetcode" }
    end

    function xml.functions.processINLINECODE(t)
        buffers.assign("inlinecode","\\type{" .. tostring(xml.text(t)) .. "}")
        context.getbuffer { "inlinecode" }
    end
\stopluacode

%%% Custom pandoc-xhtml overrides
\startxmlsetups xml:report1
  %%% Disable header section. Not used for this document.
  \xmlsetsetup{\xmldocument}
    {html|body|div|h1|h2|h3|h4|h5|h6|p|blockquote|span|em|q|b|strong|a|ul|ol|li|dl|dt|dd|hr|br|sup|sub|code|table|thead|tbody|tr|th|td}
    {xml:*}
  \xmlsetsetup{\xmldocument}
    {header|ol/li[@role='doc-endnote']}
    {}
  %%% Simpler level1 headings
  \xmlsetsetup{\xmldocument}
    {h1}
    {xml:intro}
  \xmlsetsetup{\xmldocument}
    {h2}
    {xml:h2}
  \xmlsetsetup{\xmldocument}
    {h3}
    {xml:h3}
  \xmlsetsetup{\xmldocument}
    {figure}
    {xml:figure}
  \xmlsetsetup{\xmldocument}
    {img}
    {xml:img}
  \xmlsetsetup{\xmldocument}
    {nav[@role='doc-toc']}
    {xml:toc}
  \xmlsetsetup{\xmldocument}
    {pre[contains(@data-fromfile,'yes')]}
    {xml:filecode}
  \xmlsetsetup{\xmldocument}
    {pre/code}
    {xml:code}
  \xmlsetsetup{\xmldocument}
    {!pre/code}
    {xml:inlinecode}
  \xmlsetsetup{\xmldocument}
    {div[@class="columns"]}
    {xml:columns}
  \xmlsetsetup{\xmldocument}
    {u}
    {xml:underline}
  \xmlsetsetup{\xmldocument}
    {del}
    {xml:strikethrough}
  \xmlsetsetup{\xmldocument}
    {sup}
    {xml:sup}
  \xmlsetsetup{\xmldocument}
    {jap}
    {xml:jap}
  \xmlsetsetup{\xmldocument}
    {a[@class='footnote-ref']}
    {xml:footnote:ref}
  \xmlsetsetup{\xmldocument}
    {a[@class='footnote-back']}
    {}
  \xmlsetsetup{\xmldocument}
    {div[@class='footnotes']}
    {xml:footnotes}
  \xmlsetsetup{\xmldocument}
    {page}
    {xml:pagebreak}
  \xmlsetsetup{\xmldocument}
    {table}
    {xml:table}
  \xmlsetsetup{\xmldocument}
    {thead}
    {xml:table:head}
  \xmlsetsetup{\xmldocument}
    {tbody}
    {xml:table:body}
  \xmlsetsetup{\xmldocument}
    {tr}
    {xml:table:row}
  \xmlsetsetup{\xmldocument}
    {th|td}
    {xml:table:cell}
\stopxmlsetups

\xmlregistersetup{xml:report1}

% Disable automatic pages in introductions (see bellow)
\setuphead[intro][page=no]

%%% Compact sections: Some sections may be too small to justify a pagebreak,
%%% if a section is marked with nopagebreak, skip the newpage.
\startxmlsetups xml:intro
  \xmldoifelse{#1}{//h1[contains(@class,'nopagebreak')]}
    {\page[disable]} % ignore the next request for a page, reset afterwards
    {\page} % request a pagebreak
  \intro[\xmlattribute{#1}{..//div}{id}]{\xmlflush{#1}}
\stopxmlsetups

\startxmlsetups xml:h2
  \xmldoif{#1}{//h2[contains(@class,'break')]}
    {\page}
  \section[\xmlattribute{#1}{..//div}{id}]{\xmlflush{#1}}
\stopxmlsetups

\startxmlsetups xml:h3
  \xmldoif{#1}{//h3[contains(@class,'break')]}
    {\page}
  \subsection[\xmlattribute{#1}{..//div}{id}]{\xmlflush{#1}}
\stopxmlsetups

\startxmlsetups xml:ul
  \startitemize[1, packed, broad, joinedup, nowhite, after]
    \xmlflush{#1}
  \stopitemize
\stopxmlsetups

\startxmlsetups xml:figure
  \placefigure
    {\xmlfirst{#1}{/figcaption}}
    {\xmlfirst{#1}{/img}}
\stopxmlsetups

\startxmlsetups xml:img
  \externalfigure[\xmlatt{#1}{src}][fullwidth][factor=fit]
\stopxmlsetups

\startxmlsetups xml:toc
  \completecontent[]
  \page[makeup]
\stopxmlsetups

% To use it, just add :::{.columns number="x"} in your document where :
% - x is the number of columns you want
% - use a minimum amount of 3 ':' before the div class columns and close the
%   part where you want your columns with the same number of ':'
\startxmlsetups xml:columns
  \startcolumns[n=\xmlatt{#1}{data-number}]
  	\xmlflush{#1}
  \stopcolumns
\stopxmlsetups

\startxmlsetups xml:underline
  \underbar{\xmlflush{#1}}
\stopxmlsetups

\startxmlsetups xml:strikethrough
  \overstrike{\xmlflush{#1}}
\stopxmlsetups

\startxmlsetups xml:code
  \definevimtyping[CODE][syntax=\xmlchainatt{#1}{class}]
  \pushcatcodetable
  \setcatcodetable\ctxcatcodes
  \xmlfunction{#1}{processCODE}
  \popcatcodetable
\stopxmlsetups

\startxmlsetups xml:filecode
  \definevimtyping[CODE][syntax=\xmlchainatt{#1}{class}]
  \typeCODEfile{\xmlchainatt{#1}{data-file}}
\stopxmlsetups

\startxmlsetups xml:inlinecode
  \pushcatcodetable
  \setcatcodetable\ctxcatcodes
  \xmlfunction{#1}{processINLINECODE}
  \popcatcodetable
\stopxmlsetups

\startxmlsetups xml:jap
  \usetypescriptfile[osx]
  \usebodyfont[hiragino-kaku,12pt]
  \setscript[nihongo]
  \xmltext{#1}{jap}
\stopxmlsetups

\startxmlsetups xml:pagebreak
  \page[yes,reset]
\stopxmlsetups

\startnotmode[nofootnotes]
\startxmlsetups xml:footnote:ref
  \xmlfilter{main}{section[@class='footnotes']/ol/li[@id=idstring('\xmlatt{#1}{href}')]/command(xml:footnote:set)}
\stopxmlsetups
\stopnotmode

\startxmlsetups xml:table
  \startembeddedxtable[option=stretch]
    \xmlflush{#1}
  \stopembeddedxtable
\stopxmlsetups

\startxmlsetups xml:table:head
  \startxtablehead
    \xmlflush{#1}
  \stopxtablehead
\stopxmlsetups

\startxmlsetups xml:table:body
  \startxtablebody
    \xmlflush{#1}
  \stopxtablebody
\stopxmlsetups

\startxmlsetups xml:table:row
  \startxrow
    \xmlflush{#1}
  \stopxrow
\stopxmlsetups

\startxmlsetups xml:table:cell
  \startxcell
    \xmlflush{#1}
  \stopxcell
\stopxmlsetups

\setupheads[expansion=yes]

\input ./title/title.tex
